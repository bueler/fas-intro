\documentclass[letterpaper,final,12pt,reqno]{amsart}

\usepackage[total={6.3in,9.2in},top=1.1in,left=1.1in]{geometry}

\usepackage{times,bm,bbm,empheq,fancyvrb,graphicx}
\usepackage[dvipsnames]{xcolor}
\usepackage{tikz}
\usetikzlibrary{decorations.pathreplacing}

% hyperref should be the last package we load
\usepackage[pdftex,
colorlinks=true,
plainpages=false, % only if colorlinks=true
linkcolor=blue,   % ...
citecolor=Red,    % ...
urlcolor=black    % ...
]{hyperref}

\renewcommand{\baselinestretch}{1.05}

\newtheorem{lemma}{Lemma}

\newcommand{\Matlab}{\textsc{Matlab}\xspace}
\newcommand{\eps}{\epsilon}
\newcommand{\RR}{\mathbb{R}}

\newcommand{\grad}{\nabla}
\newcommand{\Div}{\nabla\cdot}
\newcommand{\trace}{\operatorname{tr}}

\newcommand{\hbn}{\hat{\mathbf{n}}}

\newcommand{\bb}{\mathbf{b}}
\newcommand{\bbf}{\mathbf{f}}
\newcommand{\bg}{\mathbf{g}}
\newcommand{\bn}{\mathbf{n}}
\newcommand{\bu}{\mathbf{u}}
\newcommand{\bv}{\mathbf{v}}
\newcommand{\bw}{\mathbf{w}}
\newcommand{\bx}{\mathbf{x}}

\newcommand{\bV}{\mathbf{V}}
\newcommand{\bX}{\mathbf{X}}

\newcommand{\bxi}{\bm{\xi}}

\newcommand{\bzero}{\bm{0}}

\newcommand{\rhoi}{\rho_{\text{i}}}


\begin{document}
\title[Geometric multigrid for glacier modeling]{Geometric multigrid for glacier modeling: \\ A user's guide}

\author{Ed Bueler}

\begin{abstract} FIXME: two principles in introduction: mass conservation complementarity, solver optimality.  four examples in sections \ref{sec:subspace}--\ref{sec:stokes}: poisson equation from subspace decomp point of view, obstacle problem by subset decomposition, monotone multigrid for implicitly-evolving SIA geometry, Schur-complement and Vanka Newton-multigrid for fixed-geometry Glen-Stokes
\end{abstract}

\maketitle

\thispagestyle{empty}
\bigskip

\section{Introduction} \label{sec:intro}

The construction of effective numerical glacier and ice sheet models is challenging for two fundamental reasons.  First, the physics of glaciers is nonlinear, subject to incompletely-understood boundary conditions, and involves a two-phase fluid.  In fact the physics is highly-coupled in the sense that mass, momentum, and energy conservation interact in ways that are both relevant to glaciological modeling goals and which are not well-understood in the literature.  Second, the geometry of glaciers and ice sheets is complex, and in particular the fastest-flowing parts of ice sheets are near or on the geometrically nontrivial lateral boundary.  Numerical models need to perform expensive fine-mesh calculations to accomodate the complicated geometry of ice sheet and glacier boundaries.

On the other hand, since the 1980s researchers in numerial methods have developed multigrid methods to solve partial differential equations like those which describe the ice fluid in glaciers.   For simpler problems like scalar elliptic equations and the linear Stokes system, these methods are now in routine use \cite{Briggsetal2000,Bueler2021,Trottenbergetal2001}.

FIXME write it


\section{From subspace decomposition to multigrid} \label{sec:subspace}

To introduce multigrid methods we demonstrate how to solve a simple ordinary differential equation (ODE), namely the Poisson problem
\begin{equation}
- u''(x) = f(x) \quad \text{on} \quad 0 \le x \le 1, \label{eq:poisson}
\end{equation}
with boundary conditions $u(0)=u(1)=0$.  Our numerical approximation will use an unequally-spaced ``fine'' mesh of subintervals, called \emph{elements} from now on, as in Figure \ref{fig:finehats}.

\begin{figure}
\includegraphics[width=0.6\textwidth]{figs/finehats.pdf}
\caption{Hat functions $\lambda_p(x)$, for indices $p=1,\dots,m$, form a basis for the space of piecewise-linear functions $\mathcal{V}$ on the unequally-spaced fine mesh.}
\label{fig:finehats}
\end{figure}

The numerical solution $u^h(x)$ will be constructed as a linear combination of the piecewise-linear hat functions $\lambda_p(x)$ shown in Figure \ref{fig:finehats}, over all interior \emph{nodes} (points) in the mesh:
\begin{equation}
u^h(x) = \sum_{p=1}^m u^p \lambda_p(x). \label{eq:trialsolution}
\end{equation}
Note that $u^h(x)$ is continuous and piecewise-linear.  Its derivative is defined on the elements, but not generally at the nodes.

Having represented the solution, to actually solve the problem one must adjust the coefficients $u^p$.  These coefficients are formed into a column vector, $\bu=\{u^p\}$ in $\RR^m$.  We may directly use equation \eqref{eq:poisson} to construct a linear system to determine $\bu$ by a finite difference (FD) method \cite{LeVeque2007}.  However, our applications of multigrid ideas to glacier problems will be clearer if we instead use a finite element (FE) approach based on re-phrasing \eqref{eq:poisson} in \emph{weak form} using integrals.  (Accessible introductions to FE methods are in \cite{Bueler2021,Elmanetal2014,Johnson2009}.)  Note that once we state the weak form, next, then the original equation \eqref{eq:poisson} will be called the \emph{strong form}.

The weak form of \eqref{eq:poisson} arises by multiplying both sides of the equation by a \emph{test function} and integrating by parts so that only first derivatives remain.  Without committing to any mathematical detail, we suppose the exact, continuum solution $u(x)$ comes from a vector space $\mathcal{H}$ of functions which are smooth enough to allow the computations which follow.  Choosing a test function $v(x)$ from $\mathcal{H}$ also, by multiplying and integrating we find that equation \eqref{eq:poisson} implies
\begin{equation}
\int_0^1 u'(x) v'(x)\,dx = \int_0^1 f(x) v(x)\, dx.
\label{eq:weakpoisson}
\end{equation}
In this FE method one substitutes the \emph{trial} formula \eqref{eq:trialsolution} for $u^h$ into \eqref{eq:weakpoisson} to derive a linear system
\begin{equation}
A \bu = \bbf, \label{eq:linearsystem}
\end{equation}
where $A$ is an invertible $m\times m$ matrix and $\bbf$ is in $\RR^m$, by using each hat function on the fine mesh (Figure \ref{fig:finehats}) as a test function.  That is, substitution of $v=\lambda_p$ into \eqref{eq:weakpoisson} gives the $p$th equation in linear system \eqref{eq:linearsystem}.  For the right side one defines $f^p = \int_0^1 f(x) \lambda_p(x)\,dx$ and then the vector $\bbf = \{f^p\}$ in $\RR^m$.  Having done all these steps it is a simple matter to ask a computer to solve the linear system, for instance by using Gaussian elimination.

On the basis of the above simple scheme we can now build a \emph{multilevel} scheme which takes us most of the way to multigrid.  Consider the enlarged set of hat functions shown in Figure \ref{fig:coarsehats},
    $$\lambda_1(x),\lambda_2(x),\dots,\lambda_m(x),\underbrace{\lambda_{m+1}(x),\dots,\lambda_{m+s}}_{\text{coarse-mesh hat functions}}$$
FIXME DECORATE UNDERBRACE

cite for subspace decomp \cite{Xu1992}

\begin{figure}
\includegraphics[width=0.56\textwidth]{figs/coarsehats.pdf}
\medskip

\includegraphics[width=0.56\textwidth]{figs/coarsesthats.pdf}
\caption{Top: A coarse ``mesh'' is really just a set of additional hat functions $\lambda_p(x)$, for $p>m$, which spread over a greater distance.  Bottom: One may coarsen further.}
\label{fig:coarsehats}
\end{figure}


think of residual as a functional; it is represented as a vector in the fine-grid hats but it can act on the coarse grid ones too


\section{Subset decomposition for obstacle problems} \label{sec:obstacle}


cite for multigrid obstacle \cite{BrandtCryer1983,Bueler2021,GraeserKornhuber2009,Jouvetetal2013}; cite for subset decomp \cite{Tai2003}

\section{Multigrid solutions of a shallow ice sheet mass conservation problem} \label{sec:sia}

cite for glaciers as obstacle problems \cite{Bueler2016,Bueler2020,Calvoetal2002,JouvetBueler2012}

\section{Multigrid solutions of a Glen-Stokes momentum conversation problem for a glacier flowline} \label{sec:stokes}

multigrid already used for Blatter-Pattyn model \cite{BrownSmithAhmadia2013} and for hybrid \cite{Jouvetetal2013}; one goal of this section is to make these approaches more understandable; use Schur complement \cite{Bueler2021,Elmanetal2014}; compare Vanka monolithic smoother \cite{Farrelletal2019}

\small

\bigskip
\bibliography{review}
\bibliographystyle{siam}

\end{document}
